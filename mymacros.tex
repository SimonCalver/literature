

% %%%%%%%%%%% My commands
% % Derivative
% \newcommand{\D}[3]{\frac{ \text{d}^{#3} {#1} }{ \text{d} {#2}^{#3} }} 
% % Partial derivative
% \newcommand{\pD}[3]{\frac{ \partial^{#3} {#1} }{ \partial {#2}^{#3} }} 
% % 
% \newcommand{\dv}[1]{ \ \text{d} #1} 
% % Columns that fill the whole page
% \newcolumntype{Y}{>{\RaggedRight\arraybackslash}X} 
% % SIunitX setup
% \sisetup{detect-weight=true, detect-family=true}
% \sisetup{range-phrase = \text{--}}
% %% For making notes that stand out
% \definecolor{Ncolour}{rgb}{0.6,0.1,0.1}
% \newcommand{\note}[1]{{\color{Ncolour} \fontsize{14}{16.8}\selectfont \textbf{#1}}}  
% % Bessel function
% \newcommand{\besselj}[2]{J_{#1}\!\left(#2\right)}


\makeatletter
\newcommand*{\ie}{%
    \@ifnextchar{,}%
        {{\it{i.e.}}\ignorespaces} %
        {{\it{i.e.}}\ } %
}
\makeatother

\makeatletter
\newcommand*{\eg}{%
    \@ifnextchar{,}% 
        {{e.g.}\ignorespaces} %
        {{e.g.}\ } % 
}
\makeatother

% \newcolumntype{Y}{>{\RaggedRight\arraybackslash}X} 
% %%%%%%%%%%%%%%%%%%%%%%%%%%%%%%%%%%%%%%%%%%%%%%%%%%%%%%%%%%%%%%%%%%%%%%%%%%%%

% %%%%%%%%%%%%%%%%%%%%%%%%%%%%%%%%%%%%%%%%%%%%%%%%%%%%%%%%%%%%%%%%%%%%%%%%%%%%% Subfigure setup
% \floatsetup[figure]{style=plain,subcapbesideposition=top}
% %%%%%%%%%%%%%%%%%%%%%%%%%%%%%%%%%%%%%%%%%%%%%%%%%%%%%%%%%%%%%%%%%%%%%%%%%%%%

% %%%%%%%%%%%%%%%%%%%%%%%%%%%%%%%%%%%%%%%%%%%%%%%%%%%%%%%%%%%%%%%%%%%%%%%%%%%%% SIunitX setup
% \sisetup{detect-weight=true, detect-family=true}
% \sisetup{range-phrase = \text{--}}
% %%%%%%%%%%%%%%%%%%%%%%%%%%%%%%%%%%%%%%%%%%%%%%%%%%%%%%%%%%%%%%%%%%%%%%%%%%%%

% %%%%%%%%%%%%%%%%%%%%%%%%%%%%%%%%%%%%%%%%%%%%%%%%%%%%%%%%%%%%%%%%%%%%%%%%%%%%% Cleverref setup
% \crefrangelabelformat{equation}{(#3#1#4)--(#5#2#6)}
% \labelcrefformat{equation}{(#2#1#3)}

% \crefformat{chapter}{\S#2#1#3}
% \crefformat{section}{\S#2#1#3}
% \crefformat{subsection}{\S#2#1#3}
% \crefformat{subsubsection}{\S#2#1#3}
% %%%%%%%%%%%%%%%%%%%%%%%%%%%%%%%%%%%%%%%%%%%%%%%%%%%%%%%%%%%%%%%%%%%%%%%%%%%%

% %%%%%%%%%%%%%%%%%%%%%%%%%%%%%%%%%%%%%%%%%%%%%%%%%%%%%%%%%%%%%%%%%%%%%%%%%%%
% % This makes the nice widebar; it may be a bit excessive!

% \makeatletter
% \let\save@mathaccent\mathaccent

% \newcommand*\if@single[3]{%
%   \setbox0\hbox{${\mathaccent"0362{#1}}^H$}%
%   \setbox2\hbox{${\mathaccent"0362{\kern0pt#1}}^H$}%
%   \ifdim\ht0=\ht2 #3\else #2\fi
%   }
% %The bar will be moved to the right by a half of \macc@kerna, which is computed by amsmath:
% \newcommand*\rel@kern[1]{\kern#1\dimexpr\macc@kerna}
% %If there's a superscript following the bar, then no negative kern may follow the bar;
% %an additional {} makes sure that the superscript is high enough in this case:
% \newcommand*\widebar[1]{\@ifnextchar^{{\wide@bar{#1}{0}}}{\wide@bar{#1}{1}}}
% %Use a separate algorithm for single symbols:
% \newcommand*\wide@bar[2]{\if@single{#1}{\wide@bar@{#1}{#2}{1}}{\wide@bar@{#1}{#2}{2}}}
% \newcommand*\wide@bar@[3]{%
%   \begingroup
%   \def\mathaccent##1##2{%
% %Enable nesting of accents:
%     \let\mathaccent\save@mathaccent
% %If there's more than a single symbol, use the first character instead (see below):
%     \if#32 \let\macc@nucleus\first@char \fi
% %Determine the italic correction:
%     \setbox\z@\hbox{$\macc@style{\macc@nucleus}_{}$}%
%     \setbox\tw@\hbox{$\macc@style{\macc@nucleus}{}_{}$}%
%     \dimen@\wd\tw@
%     \advance\dimen@-\wd\z@
% %Now \dimen@ is the italic correction of the symbol.
%     \divide\dimen@ 3
%     \@tempdima\wd\tw@
%     \advance\@tempdima-\scriptspace
% %Now \@tempdima is the width of the symbol.
%     \divide\@tempdima 10
%     \advance\dimen@-\@tempdima
% %Now \dimen@ = (italic correction / 3) - (Breite / 10)
%     \ifdim\dimen@>\z@ \dimen@0pt\fi
% %The bar will be shortened in the case \dimen@<0 !
%     \rel@kern{0.6}\kern-\dimen@
%     \if#31
%       \overline{\rel@kern{-0.6}\kern\dimen@\macc@nucleus\rel@kern{0.4}\kern\dimen@}%
%       \advance\dimen@0.4\dimexpr\macc@kerna
% %Place the combined final kern (-\dimen@) if it is >0 or if a superscript follows:
%       \let\final@kern#2%
%       \ifdim\dimen@<\z@ \let\final@kern1\fi
%       \if\final@kern1 \kern-\dimen@\fi
%     \else
%       \overline{\rel@kern{-0.6}\kern\dimen@#1}%
%     \fi
%   }%
%   \macc@depth\@ne
%   \let\math@bgroup\@empty \let\math@egroup\macc@set@skewchar
%   \mathsurround\z@ \frozen@everymath{\mathgroup\macc@group\relax}%
%   \macc@set@skewchar\relax
%   \let\mathaccentV\macc@nested@a
% %The following initialises \macc@kerna and calls \mathaccent:
%   \if#31
%     \macc@nested@a\relax111{#1}%
%   \else
% %If the argument consists of more than one symbol, and if the first token is
% %a letter, use that letter for the computations:
%     \def\gobble@till@marker##1\endmarker{}%
%     \futurelet\first@char\gobble@till@marker#1\endmarker
%     \ifcat\noexpand\first@char A\else
%       \def\first@char{}%
%     \fi
%     \macc@nested@a\relax111{\first@char}%
%   \fi
%   \endgroup
% }
% \makeatother
% \newcommand\test[1]{%
% $#1{M}$ $#1{A}$ $#1{g}$ $#1{\beta}$ $#1{\mathcal A}^q$
% $#1{AB}^\sigma$ $#1{H}^C$ $#1{\sin z}$ $#1{W}_n$}
% %%%%%%%%%%%%%%%%%%%%%%%%%%%%%%%%%%%%%%%%%%%%%%%%%%%%%%%%%%%%%%%%%%%%%%%%%%%%